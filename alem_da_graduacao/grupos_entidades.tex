% Este arquivo .tex será incluído no arquivo .tex principal. Não é preciso
% declarar nenhum cabeçalho

\section{Grupos e Entidades da Unicamp}

\subsection{Alumni Computação Unicamp}

\begin{figure}[H]
    \centering
    \includegraphics[width=.35\textwidth]{img/alem_da_graduacao/alumni_logo.png}
\end{figure}

Alumni é o nome dado a ex-alun*s de uma universidade. Por extensão, alumni
também é o nome de organizações sem fins lucrativos motivadas em manter o
relacionamento entre a universidade e os ex-alun*s e o destes entre si, servindo
como uma rede de contatos profissionais. A comunicação entre alun*s, ex-alun*s e
professor*s proporciona um compartilhamento de experiência e informações, que
contribuem para uma diferenciação acadêmica, cultural e profissional.

O \textbf{Alumni Computação Unicamp} é uma forma de manter tod*s que passaram
pelo melhor curso de computação da América Latina conectados -- tanto alun*s
quanto ex-alun*s. Atualmente o Alumni conta com uma página no Facebook
\\(\url{fb.com/AlumniComputacaoUnicamp}), que reúne os grupos das turmas que
passaram pela Computação, com o objetivo de atingir o maior número de alun*s e
ex-alun*s.

Participe do grupo de sua turma, convide seus amigos a curtirem a página, envie
sugestões e contribua para essa ideia!

\subsection{ARU}

\subsubsection{O que é a ARU – Associação de Repúblicas da Unicamp?}

\begin{figure}[H]
    \centering
    \includegraphics[width=.35\textwidth]{img/alem_da_graduacao/aru_logo.png}
\end{figure}

Idealizada em 2008, a ARU é a entidade que representa as Repúblicas Associadas
de Barão Geraldo.  Tem como meta prestar apoio a elas, zelar pela sua segurança
e bem estar com toda a vizinhança, como também servir de espaço para discutir os
problemas apresentados em reuniões.

\begin{figure}[H]
    \centering
    \includegraphics[width=.45\textwidth]{img/alem_da_graduacao/aru_foto.jpg}
\end{figure}

Além de entregar no começo do ano um manual para *s bix*s contendo informações
sobre as Repúblicas Associadas, a ARU também realiza vários eventos, os quais
têm o objetivo de integrar e divertir *s morador*s de repúblicas. Alguns deles
são: a \textbf{Alcorrida}, a \textbf{Campanha do Agasalho}, o
\textbf{EntortaRep}, e, principalmente, o \textbf{InterReps}.

\subsubsection{Contato}

\begin{compactitemize}
    \item Site: \url{republicasunicamp.com.br}
    \item E-mail: \email{contato@republicasunicamp.com.br}
    \item Facebook: \url{fb.com/republicasunicamp}
\end{compactitemize}

\subsection{Competições de programação}

Curte programar? Nunca programou mas está gostando de MC102? Já brincou de
Olimpíada naquelas provas com direito a medalha? Vá fundo!

Para *s bix*s que ingressaram direto do ensino médio existe a \textbf{OBI --
Olimpíada Brasileira de Informática}, logo no primeiro semestre.  Anualmente,
muit*s alun*s do IC, tanto da Ciência como da Engenharia, ganham premiações
nessa competição. O site dela é \url{olimpiada.ic.unicamp.br}.  Visite-o para
mais informações.

\begin{figure}[h!]
    \centering
    \includegraphics[width=.35\textwidth]{img/alem_da_graduacao/maratona_logo.png}
\end{figure}

Mas a principal competição para alun*s de ensino superior é a \textbf{Maratona
de Programação}, que consiste de problemas mais difíceis e é feita em equipes de
três alun*s. A Unicamp tem grande tradição nessa competição, tendo levado
equipes para a competição mundial (a ACM-ICPC) em 1996, 2000, 2002, 2003, 2004,
2007, 2009, 2012 e 2013.

Além de brilhar no currículo, é muito divertido competir e a experiência obtida
nos treinamentos tem levado muit*s maratonistas a empresas como Google,
Microsoft e Facebook.

Para participar dos treinamentos para a Maratona que acontecem na Unicamp,
visite a wiki \url{www.ic.unicamp.br/~maratona/wiki}.

\subsection{Gamux}

\begin{figure}[h!]
    \centering
    \includegraphics[width=.2\textwidth]{img/alem_da_graduacao/gamux_logo.png}
\end{figure}

O \textbf{Gamux} (Grupo de Pesquisa e Desenvolvimento de Jogos da Unicamp), composto por
computeir*s e alun*s do IA (Instituto de Artes), é uma ótima oportunidade para
quem tiver curiosidade de saber como são feitos os jogos eletrônicos.

El*s costumam organizar aulas de introdução ao desenvolvimento de jogos para *s bix*s,
além de ciclos de palestras e eventos ao longo do ano para a confeção de jogos.

Fique atent* - acesse o site, a página do Facebook ou o grupo de e-mail para se manter
informado e saber mais sobre o Gamux:

\begin{compactitemize}
    \item  Site: \url{gamux.com.br}
    \item  Facebook: \url{fb.com/gamux}
    \item  Lista de discussão: \url {bit.ly/1AplfKK}
\end{compactitemize}

\subsection{DCE}

Criado em 1978, o DCE Unicamp (Diretório Central dos Estudantes) é a entidade
que representa tod*s *s estudantes de graduação da Universidade, articulando e
organizando o movimento estudantil (ME).

Cabe ao DCE representar o conjunto d*s estudantes em todos os espaços dentro e
fora da universidade, diante das mais diversas entidades (reitoria, sindicatos,
DCEs de outras universidades, centros acadêmicos, associações etc.) e movimentos
sociais.

Como articulador do ME, cabe ao DCE organizar *s estudantes na luta por uma
educação superior realmente pública, gratuita e de qualidade. Para tanto, é
papel fundamental do DCE propor, juntamente com os centros acadêmicos,
discussões políticas que extrapolem os nossos currículos e o nosso dia a dia.
Além disso, o DCE deve propor ações que vão ao encontro das reivindicações
estudantis, de forma que elas sejam levadas e cobradas da reitoria ou até mesmo
do governo.

O DCE esteve envolvido em várias conquistas dos estudantes, das quais se
destacam algumas lutas históricas: a construção da moradia estudantil; a
melhoria de estrutura para cursos noturnos, que tornou acessível para esse
período bibliotecas, xerox, laboratórios e secretarias de graduação; a reunião
semestral para avaliação de curso; o não aumento do preço do bandejão; uma
seleção mais justa para as vagas na moradia; entre diversas outras. Além disso,
o DCE teve participação em importantes lutas sociais que extrapolam o âmbito da
Unicamp, como a organização do Plebiscito contra a Alca e do Plebiscito contra o
Provão; a luta por mais verbas para a educação no estado de São Paulo; diversas
lutas pela qualidade do ensino e manutenção de direitos dos estudantes em outras
universidades como na UNIP, UNIMEP, FUPPESP etc.

No fim do ano, há eleições para definir qual a chapa que comandará a entidade no
ano seguinte, juntamente com eleições para representação discente no Consu e na
CCG. É muito importante a participação d*s alun*s nessas eleições, então estejam 
sempre ligad*s durante o ano inteiro sobre as atividades do DCE: estão
cumprindo o programa da chapa? as atividades estão atendendo às demandas estudantis?
*s integrantes da chapa do DCE estão dialogando com estudantes e com as entidades
estudantis? 
Estejam sempre antenad*s para que *s nossos representantes no DCE permaneçam 
atendando aos interesses d*s estudantes, pois assim podemos continuar os avanços 
dentro da nossa universidade.

Além disso, o DCE organiza a Calourada Integrada juntamente com os centros
acadêmicos. Elas que acontecem na sede do DCE, próxima ao Bandejão. Não deixe
de participar!

\begin{compactitemize}
    \item  Telefone: (19) 3521-7910 / (19) 3521-7042
    \item  E-mail: \email{dceunicamp@gmail.com}
    \item  Site: \url{dceunicamp.org.br}
\end{compactitemize}

\subsection{Equipe Phoenix}

\begin{figure}[h!]
    \centering
    \includegraphics[width=.35\textwidth]{img/alem_da_graduacao/phoenix_logo.png}
\end{figure}

A Equipe Phoenix de Robótica da Unicamp é composta por alunos da Mecânica,
Elétrica e Computação, e desenvolve projetos todo ano para participar de
competições nacionais como a RoboCore, disputando pela categoria de melhor robô
de combate, sumô, trekking e seguidor de linha, entre outros.

Para aquel*s interessad*s pela programação um robô, pela eletrônica das placas
de controle ou ainda pela mecânica dos robôs que resistem a impactos gigantescos
durante a guerra, a equipe realiza um processo seletivo todo começo de ano, com
inscrições através do site. Se envolva!

\begin{compactitemize}
    \item  Site: \url{phoenixunicamp.com.br}
    \item  Facebook: \url{fb.com/phoenixunicamp}
\end{compactitemize}

\subsection{LibrePlanet São Paulo}

\begin{figure}[h!]
    \centering
    \includegraphics[width=.45\textwidth]{img/alem_da_graduacao/lp-br-sp-logo.jpg}
\end{figure}

Numa sociedade controlada majoritariamente por algoritmos e com nossos dados
pessoais fluindo livremente pela Internet, surge a necessidade de lidar com
questões éticas, em especial, responder a pergunta: Quem realmente controla os
programas que você usa?  O movimento do {\bf Software Livre} busca resolver este
dilema ético, devolvendo o controle do computador aos usuários, de forma que
estes possam recuperar sua privacidade e o controle sobre sua computação.

O {\bf LibrePlanet São Paulo} é um grupo dedicado a discutir as questões que
permeiam a filosofia do Software Livre, como liberdade, \emph{hacking},
segurança e privacidade vs. vigilância estatal.  Uma parte importante do nosso
ativismo é ensinar as pessoas a reconhecerem armadilhas proprietárias na
computação. Essas armadilhas podem estar tanto em programas de computador que
executam localmente na sua máquina, quanto em serviços online que invadem nossa
privacidade e nos forçam a utilizar tecnologias prejudiciais à liberdade.  Por
isso, além de colocarmos bastante ênfase na divulgação de\\Softwares Livres,
também oferecemos alguns serviços pela internet que podem ajudar os usuários a
se livrarem dessa dependência nociva.

Desde 2013, O LibrePlanet organiza o {\bf Curso de GNU/Linux para os Bixos}.
Este curso, como o próprio nome diz, é voltado para *s bix*s como você que
ingressam nos cursos de computação, embora seja aberto também a veteran*s. O
objetivo é o ensinar o básico do sistema operacional GNU/Linux para você começar
a se virar no resto da graduação.  O curso acontece nas primeiras semanas de
aula, mas não se preocupe, você será avisad* na sala de aula por um(a) veteran* e
receberá um comunicado pelo seu e-mail do IC.

Não perca também a oportunidade de instalar o Sistema Operacional GNU/Linux no
seu computador durante o {\bf Installfest} organizado pelo LP.  O conhecimento
sobre esse sistema será extremamente útil para a sua vida como computeir*, não
só durante a faculdade!

\begin{compactitemize}
    \item  \url{libreplanetbr.org}
    \item  \url{libreplanet-br-sp@libreplanet.org}
\end{compactitemize}

\subsection{MTE}

\begin{figure}[h!]
    \centering
    \includegraphics[width=.35\textwidth]{img/alem_da_graduacao/mte_logo.png}
\end{figure}

O \textbf{MTE -- Mercado de Trabalho em Engenharia} é uma entidade estudantil
que visa colocar * alun* da Unicamp em contato mais próximo com o mercado de
trabalho e com as possibilidades que ele proporciona, mostrando as diferentes
áreas de atuação de um(a) engenheir*.

A participação no MTE desenvolve habilidades como networking, oratória,
expressão, empreendedorismo, gestão e diversas outras.

A estrutura do MTE é dividida em três pilares:

\begin{description}
    \item[Oportunidades:] responsável por atividades como visitas técnicas e
        captação de treinamentos e palestras.

    \item[Desenvolvimento:] responsável por atividades como English Meeting
        (encontros de conversação em inglês), Teia do Conhecimento (treinamentos
        dados por algum dos membros) e Ciclo de Oratória (ciclos com foco em
        melhoria de expressão).

    \item[Orientação e Carreira:] responsável pela estruturação pessoal e
        profissional dos membros realizando feedbacks, atividades de
        consultoria, de motivação, entrevista com profissionais e
        confraternizações.
\end{description}

Além da participação em um dos pilares, alguns membros participam da diretoria
administrativo-financeira e, para completar, participam da realização de dois
principais eventos: o EMC (Estudante e Mercado Conectados) que conta com visitas
técnicas e palestras e o ArenaMTE, um desafio universitário de resolução de
cases.

Para mais informações sobre o MTE e seu processo seletivo, acesse
\url{mte.org.br}.

\subsection{Rádio Muda}

Você provavelmente nunca viu nada do tipo na sua vida. Uma rádio na qual
qualquer ser humano pode fazer o seu programa tranquilamente, sem burocracias
(tendo espaço na grade de horários, lógico).

A Rádio Muda fica embaixo da caixa d'água (carinhosamente apelidada de Pau do
Zeferino) que fica perto do Teatro de Arena, bem em frente à BC (Biblioteca
Central).

Se você só quiser ouvir a muda, 88,5 MHz no seu rádio (em Barão Geraldo ou
Paulínia) ou pela Internet, através do site \url{muda.radiolivre.org}.

\subsection{Curso Exato}

O Curso Exato é um projeto da Pró-Reitoria de Extensão e Assuntos Comunitários
da Unicamp criado em 2008 por alun*s de graduação, com a finalidade de explorar
o potencial e a capacidade d*s alun*s de se expressarem, de raciocinarem
logicamente e de compreenderem o mundo que *s cerca, por meio de aulas de Língua
Portuguesa, Matemática, Física e Química.

*s professor*s do curso são alun*s de graduação e pós-graduação da universidade
e o público alvo é constituído por alun*s da rede pública de ensino com
disposição e interesse para aprender.

As aulas são realizadas no período noturno, das 19h15 às 22h30, de segunda a
quinta-feira, no campus da universidade.

\begin{compactitemize}
    \item Site: \url{bit.ly/1GajCZS}
    \item Facebook: \url{fb.com/curso.exato}
    \item E-mail: \email{curso.exato@gmail.com}
\end{compactitemize}

\subsection{Grupos Religiosos}

\subsubsection{ABU -- Aliança Bíblica Universitária}

Grupo evangélico não ligado a nenhuma denominação, organiza várias reuniões e
grupos de discussões e é filiado à Aliança Bíblica Universitária do Brasil
(\url{abub.org.br}).

\begin{compactitemize}
    \item Site: \url{abucampinas.org}
    \item E-mail: \email{contato@abucampinas.org} ou
        \email{abucamp_co@yahoogrupos.com.br}
    \item Telefone: (19) 3289-2823
\end{compactitemize}

\subsubsection{Pastoral Universitária}

Grupo católico que se reúne semanalmente para estudar textos (bíblicos ou não),
livros, documentos, aprofundar a fé e promover a integração e união de seus
participantes. A Pastoral Universitária também organiza grupos de preparação
para Primeira Comunhão e Crisma, além de duas Missas semanais e Grupos de Oração
Universitários (GOUs). As Missas são realizadas às terças (18h) e às quintas
(12h15), sempre no PB04. Os GOUs acontecem às terças (12h15) e nas quintas
(18h), também no PB04.

\begin{compactitemize}
    \item Site: \url{sites.google.com/site/pastoralunicamp}
    \item E-mail: \email{pastoralunicamp@gmail.com}
\end{compactitemize}
