% Este arquivo tex vai ser incluído no arquivo tex principal, não pe preciso
% declarar nenhum cabeçalho

\section{Comida}
\subsection{Bandejão}

Um dos momentos de glória do dia de um(a) futur* engenheir*, cientista ou
bacharel é o Bandejão. É a hora de intensas e indiscutíveis emoções. Caso sua
salada corra sobre a mesa, mantenha-se calm*. Evite discussões, jamais tente
descobrir o sabor do suco pelo paladar (caju ou manga?). É mais cômodo ler no
cardápio do dia. Uma dica: para cortar o bife faça muita força e, quando começar
a amolecer, pare, você chegou na bandeja.

Falando sério agora: o RU (Restaurante Universitário), ou Bandejão, ou ainda
Bandeco, fica ao lado da Biblioteca Central, bem em frente ao PB (Prédio Básico,
ou Ciclo Básico II) e, a menos que você não queira economizar uma boa grana com
comida, vai ser o lugar onde você vai estar na maioria dos seus horários de
almoço. Com o tempo, você vai ver que o Bandeco é o ``coração da Unicamp''. É o
local de você se encontrar com *s amig*s (combinando ou não antes), contar os
micos nas aulas, jogar conversa fora e falar mal da comida, que nem é tão ruim
assim como muitos dizem. Sem dúvida, é o melhor custo-benefício da Unicamp. Por
R\$2,00, você tem direito a arroz, feijão, pão, salada, proteína de soja, suco e
café à vontade. A carne e a sobremesa tem que dar uma choradinha para * ti*
para poder repetir, mas geralmente dá certo.

\begin{figure}[h!]
    \centering
    \includegraphics[width=.45\textwidth]{img/barao/bandeco.jpg}
\end{figure}

Existe também o RA (Restaurante Administrativo, não confundir com registro
acadêmico), também conhecido como Prateco, pelo fato de a comida ser servida em
pratos e não em bandejas. Fica atrás da Faculdade de Engenharia Elétrica e de
Computação (FEEC), perto do prédio da Engenharia Básica. Tem algumas diferenças
em relação ao Bandeco: o espaço físico é bem menor, por exemplo. No RA você
mesmo se serve, apesar de a carne ser servida pel* ti* que trabalha
lá. Dependendo de onde você vai ter aula antes ou depois do almoço, é mais
negócio almoçar no RA. Para poder usar o Bandeco e o RA, você deve estar com o
seu Cartão Universitário (também chamado de RA) carregado.

Além do RA, há o novo restaurante universitário, conhecido como RS (Restaurante
Universitário da Saturnino). Ele é localizado próximo ao IC-3 e ao prédio azul
da Civil. Semelhante ao RA, você come em pratos, *s ti*s servem a carne e o
resto é self-service. As vantagens são que o ambiente é menos claustrofóbico, há
mais lugares e a localização o torna muito prático para os computeiros. Porém,
só está aberto no período do almoço. Desconsiderando a localização, o RS com
certeza é a melhor opção no almoço.

\subsubsection{Cardápio vegetariano}

Um cardápio ovo-lacto-vegetariano está sendo oferecido desde o fim de 2013. Ele
está disponível no RS (ou seja, somente no almoço), numa fila separada. Agora, é
possível saborear pratos tais como abobrinha agridoce, hambúrguer de soja,
legumes com molho branco, torta de aveia, entre outros.

No período de férias, quando o RS está fechado, o cardápio vegetariano passa a
ser servido no RU. Não há fila separada, você precisa pedir por ele em vez da
carne.

\subsubsection{RS na janta}

O RS, inaugurado no fim de 2012, operou somente no período do almoço, até o
fechamento desta edição do Manual. Porém, por conta de uma demanda do CACo,
juntamente com outros centros acadêmicos e o DCE, o que levou a um
abaixo-assinado e então a uma petição, foi autorizado o funcionamento do RS no
período do jantar.

\subsubsection{Café da manhã}

Em março de 2016 houve o início das operações do RU no período da manhã,
atendendo a uma demanda d*s estudantes. De acordo com o reitor Tadeu Jorge, a
Unicamp investiu cerca de R\$ 70 mil para equipar os restaurantes para oferecer
o café da manhã. O período é das 7h às 8h30.

O preço é apenas R\$ 1,00. O cardápio consiste de café, leite, pão, manteiga,
geleia e fruta, com algumas variações.

\subsubsection{Como funciona o esquema de carregar o cartão?}

Simples. Você vai à uma das máquinas que existem na entrada dos restaurantes e
faz, por exemplo, um depósito de R\$20,00 para 10 créditos. As máquinas aceitam
apenas notas, e sem troco. Outra maneira de colocar créditos é fazer um depósito
na conta do Bandeco no Santander (Ag.: 207 / Conta: 43.010.009-2) ou no Banco do
Brasil (Ag.: 4203-X / Conta: 66.315-8) e depois carregar o seu cartão, na
Prefeitura do Campus (próximo à Reitoria), apresentando o comprovante de
depósito. Não são aceitos comprovantes de pagamento de entrega de envelope ou
via internet.

Os restaurantes funcionam de segunda a sexta, nos seguintes horários:

\begin{itemize}
\item RU, das 7h às 8h30 (café da manhã), das 10h30 às 14h (almoço) e das 17h30
 às 19h45 (jantar).
\item RA, das 11h30 às 14h (almoço) e das 17h30 às 19h (jantar).
\item RS, das 11h30 às 14h (almoço) e das 17h30 às 19h (jantar).
\end{itemize}

Em períodos especiais, como fim de ano, os restaurantes podem funcionar em
horários reduzidos, alguns não abrem, ou então fecham completamente, fique de
olho nos emails e informes que você recebe. Geralmente o RU é o que permanece
aberto.

Para saber previamente o cardápio do Bandejão, acesse o site da Prefeitura do
Campus (\url{prefeitura.unicamp.br}) ou o GDE (\url{gde.ir}).

Para Android e iOS, está disponível o aplicativo Unicamp Serviços, do CCUEC, que
informa cardápio e saldo no seu cartão, entre outros.

\subsection{Outros lugares para as refeições}

Algumas opções dentro da Unicamp são:

\begin{itemize}
\item Cantina da Física: tem self-service.
\item Cantina da FEA/FEM: próxima a FEEC, também tem self-service no almoço. Pra
  quem não gosta de café, fica a dica da latinha de Red Bull a R\$ 5,99.
\item Gatti: Se você é vegetariano, é uma boa dica. Fica do lado do IC-2, na
  Cênicas/Dança.
\item Cantina da Biologia: Tem self-service e marmita.
\item Cantina da Química:  Tem prato-feito, abre também aos sábados. O melhor é
  o pão de queijo em dobro às quartas e quintas!
\item Cantina da Educação: Tem self-service.
\end {itemize}

Fora da Unicamp:

\begin{itemize}
\item Terraço: próximo ao balão da Av. 1, vende marmitex e tem self-service a um
  preço bom, além de churrasco às terças, quintas e sábados.
\item Bardana: Um pouco mais acima na Av. 1, com a fachada toda verde. Está na
  mesma faixa de preço do Terraço, e costuma ser considerado bem melhor; tem
  churrasco de carne bovina meio que dia-sim-dia-não, e nos outros dias é de
  frango. No jantar, há pratos à la carte, e pizza.
\item Pepe Loco: serve comida mexicana no estilo fast-food. Porém costuma ser
  bem caro pela qualidade e quantidade que oferece.
\item Aulus: na Av. 2, próximo ao balão, que é o mais caro dos citados aqui, mas
  é muito bom (e bonito). O cardápio geralmente inclui peixes e frutos do mar, e
  tem churrasco todo dia.
\item Campus Grill: em frente à guarita do HC, comida boa a um preço um tanto
  alto.
\item Outros: Na frente da reitoria há o Del Sol, o Ginza e o Moriá.  O Del Sol
  serve comida por quilo, sendo parecido (em preço e pratos) com o Bardana,
  enquanto que o Ginza serve a la carte com preços bons e o Moriá serve pratos
  feitos a preços mais baratos.
\end{itemize}

\subsection{Lanches e sucos}

Tá de tarde, bateu fome, quer comer um lanche (hamburger, pão-na-chapa, queijo
quente, x-salada, croissant, qualquer coisa do gênero)? Quase todas as cantinas
da Unicamp servem lanches. Algumas muito boas são a cantina da Mecânica e a
lanchonete da Economia.

Se você estiver no IC quando bater a fome, as opções mais próximas são a cantina
da Economia e a do Gatti. Perto da FEEC existem a Padaria da FEA e a Cantina da
Mecânica. Já nas redondezas do CB existem as cantinas da Física, Química e
Biologia.

Quase todas as cantinas servem salgados prontos, lanches naturais, doces e
demais coisas do gênero.

Para sucos, tem um lugar muito bom: a famosíssima banca de sucos do CB, que tem
milhões de sucos, vende frutas e também salgados.  Se você precisa almoçar
rápido, provavelmente sua escolha será salgado + vitamina na banca de sucos do
CB. Todo dia a banca de sucos do CB tem um sabor na oferta, que é ótimo pra sair
do tradicional suco de laranja.

Nas quartas e quintas há uma feira no centro da praça do CB, na qual há opções bem
variadas, desde pastéis a comida japonesa, embora geralmente mais caras que as
cantinas. Uma opção interessante é o Porqueta, que serve costelinhas de porco
assadas e sanduiches muito bem feitos, mas é um pouco caro.

Açaí: Se por algum motivo você tiver de andar até o quarteirão de salas de aula
da medicina, estiver cansado, e quiser um açaí, o da cantina de lá é caro e
inacreditavelmente zoado. O açaí com melhor custo-benefício da Unicamp é o da
feirinha, mas só existe as quartas e quintas, então aproveite nesses dias para
matar sua vontade.


\subsection{Padarias e café da manhã}

A cantina da Mecânica abre bem cedo e serve o bom pingado + pão na chapa
matinal.

\subsubsection{Padaria Alemã}

A Padaria Alemã, na Av. 1, próxima ao Dalben, serve uma bandeja de café da manhã
com suco, café-com-leite/chocolate, croissant, mamão, bolo, pão francês,
torradas, manteiga e geleia. Ainda há a possibilidade de fazer trocas como: suco
por chocolate, croissant por dois pães-na-chapa, mamão por banana, coisas do
gênero.
\begin{figure}[h!]
    \centering
    \includegraphics[width=.45\textwidth]{img/barao/padaria.jpg}
\end{figure}

Também são servidos lanches gigantescos, com muitas opções de recheio, por um
preço relativamente barato, então tenha alguém para dividir se sua fome não for
muita (acredite, meio lanche já serve como um almoço completo). Dependendo do
recheio, a pizza é muito barata, também, embora el*s não façam entrega. É bom
lembrar que el*s servem café da manhã das 7h até às 13h (mas a padaria só fecha
às 22h), então é uma boa pedida para se você não quiser almoçar ou para sábado e
domingo, acordar tarde e tomar um café da manhã para valer pelo almoço.

\subsubsection{Paneteria Di Capri}

Na Estrada da Rhodia, próximo à entrada da Cidade Universitária II, há a
Paneteria Di Capri, que tem um pão francês muito bom (a um preço legal) e também
muita variedade (incluindo tortas e lanches).

Além disso você também pode tomar seu café da manhã lá, pois como quase toda
padaria el*s também oferecem um cardápio bom para logo cedo. Se você estiver com
bastante apetite, de sexta a domingo el*s servem um buffet de café da manhã com
muitas opções e a um preço fixo (em torno de R\$12).

Na hora do almoço também são preparados alguns pratos (para comer no local e
para levar) e também há um esquema onde você pede um grelhado e tem acesso livre
a um balcão com saladas e outras coisas, como petiscos. À noite el*s servem
pizzas e também há o esquema do grelhado, exceto no inverno, quando el*s servem
um buffet de sopas.

\subsubsection{Padaria da FEA}

Já se você está na Unicamp e quer uma padaria, a dica é a Padaria da FEA (fica
próxima à Cantina da Mecânica). Lá el*s têm pães, doces e bolos. Com uma
diferença: há produtos especiais, como pão de queijo com linhaça ou alho e pão
francês com soja. Mas não se assuste: por mais estranho que pareçam, os produtos
de lá são muito bons! E não deixe para ir lá depois das aulas, pois a Padaria da
FEA fecha às 17h.

\subsection{E no fim de semana?}

Nos fins de semana, nem o Bandex nem quase nenhuma cantina da Unicamp abrem (e
só no sábado, se abrirem). Você vai ter que se virar fora da Unicamp.

Na Av. 1 e proximidades tem o Terraço, o Bardana e a Padaria Alemã já citados,
além de vários restaurantes próximos à Alemã.

Na Av. 2 tem o Aulus, mais caro no sábado que durante a semana; domingo, então,
mais ainda, mas costuma ter camarão à milanesa; porém a marmita tem opções de
carne e acompanhamentos (peça patachu), é grande e não é cara como o
self-service, R\$ 13,75

Um pouco mais pra cima na avenida, há o Yaki-Ten, que serve comida chinesa por
quilo e japonesa por pessoa.  Logo mais abaixo há o Ilha do Barão.

No centro de Barão não faltam opções. Tem (indo da entrada de Barão pela Estrada
da Rhodia) o Estância Grill, o Barão da Picanha, o Gordão Burguers, o Solar dos
Pampas, o Estância d'Oliveira, o Vila Santo Antonio, o Ki-Pizza, o restaurante
Baroneza, o Salsinha e Cebolinha, o Alabama, o Pão de Açúcar, o McDonald's, o
Burger King e o Subway no Tilli Center. Na frente do Pague Menos tem o Lótus,
vegetariano (não vegano), barato e bom.

Na Av.  Santa Isabel e adjacências tem o Cronópio (numa rua paralela à Santa
Isabel), o Frangonete (próximo ao Santander), o HotDog Central e as Pizzarias
Sapore Pizza e Pizza Fiori. Perto da moradia tem a Tonha (Canto do Acarajé), o
Kalunga Lanches e o famoso dogão da moradia.

Por fim, próximo à padaria Di Capri, há alguns restaurantes mais caros, como a
Romana (serviço parecido com o da Di Capri, porém um bocado mais cara), Pizzaria
Gregória, o TBONE (el*s também tem marmitex), o Greg Burgers (o hambúrguer e o
milk-shake são excelentes), o Tábua dos Mares e o Morena-flor.

\begin{figure}[h!]
    \centering
    \includegraphics[width=.45\textwidth]{img/barao/pizza.jpg}
\end{figure}

\subsubsection{Alguns telefones:}

\begin{itemize}
    \item   \textbf{Restaurante Baronesa}
        \\Telefone: (19) 3289-9087
        \\Endereço: Rua Benedito Alves Aranha, 44
        \\Site: \url{restaurantebaronesa.com.br}

    \item   \textbf{China In Box} (Faz entrega em Barão)
        \\Telefone: (19) 3254-5601
        \\Endereço: Rua Romualdo Andreazzi, 333
        \\Site: \url{chinainbox.com.br}

    \item   \textbf{TBONE Steak Bar}
        \\Telefone: (19) 3289-0485
        \\Endereço: Rua Maria Tereza Dias da Silva, 700

    \item   \textbf{Ginza Bar}
        \\Telefone: (19) 3289-9281
        \\Endereço: Rua Roxo Moreira, 1768

    \item   \textbf{Bardana}
        \\Telefone: (19) 3289-9073
        \\Endereço: Av. Dr. Romeu Tortima, 1500
        \\Site: \url{buffetbardana.com.br}

    \item   \textbf{Terraço}
        \\Telefone: (19) 3289-7920
        \\Endereço: Rua Roxo Moreira, 1344

    \item   \textbf{Pastelaria Oba-Oba}
        \\Telefone: (19) 3249-1908
        \\Endereço: Rua Benedito Alves Aranha, 115

    \item   \textbf{Pizza Mais}
        \\Telefone: (19) 3289-0320 / (19) 3289-2754

    \item   \textbf{Barão das Pizzas}
        \\Telefone: (19) 3249-1630
        \\Endereço: Rua Jerônimo Pattaro, 351
        \\Site: \url{baraodaspizzas.com.br}

    \item   \textbf{Pizza Fiori}
        \\Telefone: (19) 3289-3514
        \\Endereço: Av. Santa Isabel, 405
        \\Site: \url{pizzafiori.com.br}

    \item   \textbf{Ki-Pizza}
        \\Telefone: (19) 3289-0863
        \\Endereço: Rua Horácio Leonardi, 76

    \item   \textbf{Super Mega Pizza}
        \\Endereço: Rua Francisca Resende Merciai, 125B
        \\Telefone: (19) 3288-0606 / (19) 3288-0608

    \item   \textbf{NADOG'S -- Hot Dog do Nado}
        \\Telefone: (19) 3029-2270

    \item   \textbf{Casa da Moqueca} (prato mais caro, mas serve duas pessoas)
        \\Telefone: (19) 3289-3131
        \\Site: \url{casadamoqueca.com.br}

% \end{itemize}

% \subsection{E à noite?}

% \begin{itemize}
    \item   \textbf{Hot-dog Independência:}
        \\Telefone: (19) 3289-8805
        \\Endereço: Rua Angela Signol Grigol, 742
        \\\\
        Tem vários tipos de hot-dogs (com catupiry, com cheddar, com frango
        {\dots}) e tem preços menores que os do Rod Burguers. O único problema é
        que el*s cobram taxa de entrega para um lanche e fecham à meia-noite.

    \item   \textbf{Kalunga Lanches:}
        \\Telefone: (19) 3289-5236
        \\Endereço: Rua Sebastião Bonomi, 40
        \\\\
        Perto da moradia, el*s não entregam, mas ficam abertos até altas horas.
        Destaque para o caldinho de feijão. Obs: o lugar é limpo e bom.

    \item   \textbf{Barão Hamburgueria:}
        \\Telefone: (19) 3289-9753
        \\Endereço: Av. Albino J. B. de Oliveira, 476
        \\\\
        Rua Localizada na entrada de Barão Geraldo servem lanches parecidos com
        os do Mega Sandubão, lá el*s dão outro tipo de maionese e em geral os
        preços são tão caros quanto do Mega Sandubão. Também entregam até meia
        noite.

    \item   \textbf{Lanchão \& Cia:}
        \\Telefone: (19) 3289-3665
        \\Endereço: Av. Albino J. B. de Oliveira, 1214
        \\Site: \url{lanchao.com.br}
        \\\\
        Um dos melhores lanches de Campinas (quiçá o melhor). Os lanches
        geralmente são grandes e muito bons, e os preços são compatíveis com a
        qualidade e quantidade. El*s servem no carro se você preferir, com uma
        bandeja que fica presa no vidro. Fica no centro de Barão Geraldo,
        proximo ao Santander e Pão de Açúcar. Destaque para a batata frita,
        feita de uma forma muito diferente, extremamente crocante e quase
        cremosa por dentro.

    \item   \textbf{Burger King:}
        \\Endereço: Av. Albino J. B. de Oliveira, 1000
        \\\\
        Aberto das 10h às 22h

    \item   \textbf{Ponto 1:}
        \\Telefone: (19) 3289-2378
        \\Endereço: Rua Eduardo Modesto, 54
        \\Site: \url{ponto1bar.com}

      \item \textbf{Sapore Pizza:}
        \\Telefone: (19) 3289-0228
        \\Endereço: Av. Santa Isabel, 326
        \\Site: \url{saporepizzaria.com.br}
        \\\\
        Para quando você estiver com pelo menos mais um amigo para rachar a
        pizza, acaba sendo uma boa pedida. Geralmente as pizzas de mussarela e
        de calabresa estão com preços bem acessíveis. Além de pizzas, el*s fazem
        esfihas e batata rechada. El*s entregam até 23h.

        A Sapore também tem self-service no almoço, R\$ 15,50 por pessoa durante
        a semana e um pouco mais aos fins de semana, e marmita pra retirar no
        local, R\$ 36 o quilo pra você montar sua marmita com as coisas do
        self-service, ou R\$ 11 a marmita pronta, há várias opções de carnes.

    \item   \textbf{McDonald's:}
        \\Telefone: (19) 3289-0318
        \\Endereço: Av. Albino J. B. de Oliveira, 1430
        \\\\
        Dispensa apresentações. Entregas das 11h às 23h. Costuma ficar aberto de
        madrugada, até as 4 da manhã.

    \item   \textbf{Barraquinhas:}
        \\Há várias barraquinhas de hot-dog no centro de Barão e perto da
        moradia. Destaque para o dog do terminal, o Hot Dog Central, o Pedrogue
        e o dogão da moradia. Se você quiser um lanche, uma boa pedida é o Star
        Tresh (Raimundão ou Guarujá, chame como você quiser), que fica perto do
        balão da Avenida 2 e costuma ficar aberto até altas horas. Perto da
        Unicamp, ao lado do posto Ipiranga que fica na avenida 1 também tem um
        dog prensado muito bom e barato.
\end{itemize}

\subsection{Marmitex}

Entrega em casa. Bom e barato.

\begin{itemize}
    \item   \textbf{Alabama}
        \\Telefone: (19) 3249-0146

    \item   \textbf{tia Rita}
        \\Telefone: (19) 3249-2899

    \item   \textbf{Hailton}
        \\Telefone: (19) 3249-0153
\end{itemize}

Obs: A Sapore Pizza também entrega Marmitex.

\begin{figure}[h!]
    \centering
    \includegraphics[width=.45\textwidth]{img/barao/marmitex.jpg}
\end{figure}

\subsection{Bares, lanchonetes e restaurantes}

\begin{itemize}
    \item   \textbf{Açaizeiro Brasil:} Serve um açaí muito bom e vários tipos de
        comidas mais leves, como lanches naturais, crepes e saladas, além de
        vários sucos. O preço não é caro e a comida é boa.
        \\Endereço: Av. Santa Isabel, 518
        \\Telefone: (19) 3365-6555 %TODO: verificar número

    \item   \textbf{Aulus VideoBar \& Restaurant:} A comida é muito boa, porém
        cara, especialmente no final de semana. A exceção fica no preço do
        marmitex, apenas R\$ 12,50. O ambiente do restaurante é muito diferenciado,
        com bicicletas e ferroramas no teto, por exemplo.
        \\Endereço: Av. Prof. Atílio Martini, 939
        \\Telefone: (19) 3289-4453
        \\Site: \url{aulus.com.br}

    \item   \textbf{Bagdá Café -- Bar \& Esfiharia:} Esfihas boas, mas um pouco
        caras. Entregam em Barão (cardápio no site), mas em horários de pico
        costumam demorar um pouco. A música ambiente inclui música ao vivo e
        ritmos variados, desde a MPB ao Blues.
        \\Endereço: Av. Santa Isabel, 233
        \\Telefone: (19) 3289-0541 / (19) 3289-1842
        \\Site: \url{bagdacafe.com.br}

\begin{figure}[h!]
    \centering
    \includegraphics[width=.45\textwidth]{img/barao/bar.jpg}
\end{figure}

    \item   \textbf{Bar do Coxinha:} Famoso pela coxinha (realmente boa), vale a
        pena ir lá, mas é relativamente caro. Localiza-se perto da avenida Santa
        Isabel, na rua da Sapore Pizza.

    \item   \textbf{Bar do Jair:} Outro lugar famoso pela coxinha: só que esta é
        de carne seca. Fica relativamente perto da Moradia.
        \\Endereço: Rua Eduardo Modesto, 212

    \item   \textbf{Barão da picanha:} Churrascaria rodízio localizada na
        avenida Albino José Barbosa de Oliveira, logo na entrada de Barão.

    \item   \textbf{Batataria Suiça:} Do lado do Mega Sandubão, serve batatas
        recheadas bem diferentes. É um pouco caro, mas vale a pena conferir. Uma
        dica é que às terças-feiras você compra uma batata, mas recebe duas.
        \\Endereço: Estrada da Rhodia -- Praça José Geraldi, a 50m do posto Esso
        \\Telefone: (19) 3201-1174
        \\Site: \url{battataria.com.br}

    \item   \textbf{Boi Falô:} O restaurante é um rancho, com comida típica do
        interior. É excelente, mas um pouco caro (cerca de R\$30,00 por pessoa),
        um lugar perfeito para levar seus pais quando el*s vêm te visitar (e
        pagam o almoço!). Abre apenas nos almoços de sábado e domingo.
        \\Endereço: Rua do Sol, 600
        \\Telefone: (19) 3289-6671 / (19) 3287-6342%TODO: verificar número

    \item   \textbf{Cachaçaria Água Doce:} Localizada na avenida 1, é um lugar
        frequentado por pessoas mais velhas, ótimo para comida e bebida (pinga,
        especialmente), mas é bem caro.

    \item   \textbf{Casa São Jorge:} Música ao vivo todas as noites, com boa
        variedade. Localiza-se na rua Santa Isabel, mais ou menos perto da
        moradia.

    \item   \textbf{Empório Nono:} Caro, tem um chopp muito bem tirado e os
        melhores petiscos de Campinas. Localiza-se na avenida Albino José
        Barbosa de Oliveira, quase em frente ao terminal.
        \\Site: \url{emporiodonono.com.br}

    \item   \textbf{Estância Grill:} Logo na entrada de Barão. Tem rodízios de
        carne e de pizza à noite.
        \\Endereço: Av. Albino J. B. de Oliveira, 271
        \\Site: \url{www.estanciacampinas.com.br}
        \\Telefone: (19) 3289-8697 / (19) 3289-6055 / (19) 3289-1511

    \item   \textbf{Domino's Pizzaria:} Famosa rede de pizzarias. Tem pizza
        em dobro às terças.
        \\Endereço: Av. Albino J. B. de Oliveira, 1453
        \\Site: \url{http://www.dominos.com.br}
        \\Telefone: (19) 3368-7557

    \item  \textbf{Raizes Zen:} Restaurante vegetariano/vegano, com um bom
        preço.
        \\Endereço: Rua Antonio Pierozzi, 94
        \\Site: \url{raizeszen.com.br/barao_geraldo/index}
        \\Telefone: (19) 3305-2667 / (19) 3288-0531

    \item  \textbf{V-Burguer:} Hamburgueria vegetariana.
        \\Endereço: Av. Dr. Romeu Tortima, 104
        \\Site: \url{vburguer.com}
        \\Telefone: (19) 3325-4799

    \item  \textbf{Ala Verde:} Lanchonete vegetariana, com delivery.
        \\Endereço: Rua José Pugliesi Filho, 451
        \\Site: \url{alaverde.com.br}
        \\Telefone: (19) 3287.7165 / (19) 99461.5938

    \item  \textbf{Namaste Salad:} Restaurante com comida caseira.
        \\Endereço: R. José Martins, 751
        \\Site: \url{namastesalad.com.br}
        \\Telefone: (19) 3289-4178

	\item	\textbf{De La Rua:} Food Truck especializado em burritos.
		\\Endereço: Av. Romeu Tortima, em frente ao Wizard.

    \item  \textbf{Fernando's:} No centro de Barão, perto do Banespa, serve
        cerveja e lanches baratos e muito bons principalmente porque vêm
        acompanhados de uma porção pequena de fritas! Um lugar simples mas muito
        limpo e agradável principalmente em relação ao atendimento. Fecha as 23h
        se segunda a quinta e sábado, tem música ao vivo na sexta e por enquando
        ainda não abre nos domingos.

    \item   \textbf{Fran's Café:} Cafeteria. Vende lanches, cafés, doces,
        salgados e bebidas (quentes ou geladas). Fazem também cafés da manhã.
        Mas é um pouco caro.
        \\Endereço: Av. Albino J. B. de Oliveira, 1600

    \item   \textbf{Greg Burguers:} Uma lanchonete muito boa, mas também muito
        cara.  Uma das especialidades lá é o milk-shake (realmente muito bom).
        Fica na estrada da Rhodia (na esquina da Paneteria Di Capri). Só
        funciona à noite, de terça a domingo.
        \\Endereço: Rua Maria Tereza Dias da Silva, 664
        \\Telefone: (19) 3289-6400
        \\Site: \url{gregburgers.com.br}

    \item    \textbf{Bronco Burger:} Lanchonete conhecida por colocar sua marca
        no pão dos sanduíches.
        \\Endereço: Rua Agostinho Pattaro, 199
        \\Site: \url{broncoburger.com.br}

    \item    \textbf{Roots Burger:} Food truck especializado em sanduíches de
        costela.
        \\Endereço: Av. Santa Isabel, 369
        \\Telefone: (19) 98131-0660

    \item    \textbf{Du Lanches/Celso Lanches:} Carrinho de lanches com um otimo
        preço, com opções também vegetarianas.
        \\Endereço: Av. Albino J. B. de Oliveira, perto do Terminal de Barão.

\end{itemize}

\begin{figure}[h!]
    \centering
    \includegraphics[width=.45\textwidth]{img/barao/burger.jpg}
\end{figure}

\begin{itemize}
    \item   \textbf{La Salamandra:} Restaurante mexicano, localizado na Av. 2,
        perto do Yaki-Ten. Comida boa e preço compatível. Tabém é um ótimo lugar
        para se levar os pais quando el*s vêm visitar.
        \\Endereço: Av. Prof. Atílio Martini, 152
        \\Telefone: (19) 3289-2011
        \\Site: \url{lasalamandratexmex.com.br}

    \item    \textbf{Gua.co:} Restaurante mexicano.
        \\Endereço: Av. Albino J. B. de Oliveira, 1615
        \\Site: \url{guacamolecompany.com.br}
        \\Telefone: (19) 3365-2558

    \item   \textbf{Makis Place:} Temakeria próxima ao terminal.
        \\Endereço: Av. Albino J. B. de Oliveira, 976
        \\Telefone: (19) 3367-3077
        \\Site: \url{makis.com.br}

    \item   \textbf{Mega Sandubão:}
      Antigo Ponto Final. Lanchonete localizada na estrada da Rhodia
      (continuação da avenida Albino José de Oliveira) que entrega
      lanches até a meia noite. Muitos gostam bastante dessa
      lanchonete pela famosa maionese temperada que el*s servem, não
      se esqueça de pedir quando for comprar lanches. À noite serve
      cerveja a um bom preço. Localiza-se na estrada da Rhodia.
        \\Endereço: Av. Albino J. B. de Oliveira, 2287
        \\Telefone: (19) 3288-0204

    \item   \textbf{Quintal do Neto:} No alto da avenida 1, perto do balão de
        entrada em barão geraldo, tem cerveja a preços razoáveis, salgados
        (coxinha e quibe) grandes, e mesas de sinuca (de ficha e por hora).
        \\Endereço: Av. Dr Romeu Tórtima, 104

    \item   \textbf{Rudá:} Localizado na Santa Isabel, bar com música ambiente.

    \item   \textbf{Solar dos Pampas:} Buffet excelente. Custa R\$ 24,00 apenas
        a comida e R\$ 34,00 com refrigerante e suco incluídos (o tanto que você
        conseguir beber). Fazem um esquema no aniversário das pessoas que sai
        por R\$ 18,00 com rodízio, cerveja, refrigerante, buffet, sorvete e
        pinga à vontade. Ao lado do Estância d'Oliveira.
        \\Endereço: Av. Dr. Romeu Tortima, 165
        \\Telefone: (19) 3289-1484 / (19) 3289-7869

    \item   \textbf{Star Clean:} É o bar mais próximo à Unicamp, e por isso está
        sempre cheio. Principal ponto de encontro depois da aula e tem um bom
        preço.

    \item   \textbf{Subway:} Lanchonete. Vende dos mais variados tipos de
        lanches.  Lanches muito bons, e não tão caros. Localiza-se no Tilli
        Center (avenida Albino José Barbosa de Oliveira, 1556, esquina com a
        avenida 2). Do lado do Subway tem um caixa 24 horas que trabalha com os
        principais bancos. O Subway faz entregas em algumas regiões de Barão.
        \\Telefone: (19) 3201-8411 / (19) 3201-8410
        \\Site: \url{subdelivery.com.br}

      \item \textbf{Temakeria Barão Geraldo:} Lugar relativamente novo, meio
        caro. Vende só temaki e bebidas. O horário de funcionamento é bastante
        conveniente.
        \\Endereço: Av. Dr. Romeu Tortima, 1259
        \\Telefone: (19) 3289-0802
        \\Horário de funcionamento: domingo a terça das 11h30 às 0h, quarta a
        sábado das 11h30 às 6h
        \\Site: \url{tmkr.com.br}

    \item   \textbf{Estância d'Oliveira:} Antigo Universo das Massas. Rodízio de
        massas perto do Terminal. Bom e não é caro. De domingo à noite é o
        horário mais barato e dá pra encher bem o bucho de massa. Depois de ir
        até lá, você não vai querer saber de comer massas por um bom tempo.
        \\Endereço: Av. Albino J. B. de Oliveira, 576
        \\Telefone: (19) 3289-5369
        \\Site: \url{estanciadoliveira.com}

    \item   \textbf{Vila Ré - Pizza:} Pizzaria próxima do terminal e do
        supermercado Dalben. Tem alguns sabores diferentes, as pizzas são boas e
        o preço não é alto. Possui serviço de entrega das 18h às 23h.
        \\Endereço: Av. Albino J. B. de Oliveira, 658
        \\Telefone: (19) 3289-0321

    \item   \textbf{Bar do Zé:} O pub tem apresentações ao vivo todas as
        semanas.  Localiza-se também na avenida Albino José de Oliveira, bem em
        frente ao Pão de Açúcar.
        \\Telefone: (19) 3289 3159

    \item   \textbf{Echos Studio Bar:} Um bar relativamente novo, possui
        apresentaçes ao vivo direto, que costumam ser de Rock, Blues ou Jazz.
        Fica entre a Santa Isabel e Albino J. B. de Oliveira.
        \\Endereço: Rua Agostinha Pátaro, 54
        \\Site: \url{echos.mus.br/studiobar}
        \\Telefone: (19) 3201-8900

    \item   \textbf{Marambar:} Possui bebidas, lanches, sucos e porções a preços
        razoáveis. Ambiente agradável, ao ar livre, muito próximo da Unicamp.
        Bastante frequentado por computeir*s e engenheir*s em geral, além de
        muita gente de outros cursos. Funciona de segunda a sexta, das 7h30 até
        as 2h30, e aos sábados, das 9h até a 1h.
        \\Endereço: Av. Dr. Romeu Tortima, 1538 (próximo ao balão da avenida 1)

\end{itemize}
