% Este arquivo .tex será incluído no arquivo .tex principal. Não é preciso
% declarar nenhum cabeçalho
\section{Mensagem da FEEC}
Alunos, prezadas alunas, ingressantes no curso de Engenharia de Computação,

Parabéns pela sua conquista, é com muita alegria que lhes acolhemos na UNICAMP
e, em especial, na Faculdade de Engenharia Elétrica e de Computação (FEEC). A
vida universitária é uma fase muito especial de nossas vidas: alguns poucos
anos, tão intensos quanto breves, mas que costumam ser determinantes para nossas
escolhas, para que forjemos nosso modo de agir, pensar e ver o mundo. É um
percurso no qual cabe a vocês mesmos descobrir as nuances, as alternativas,
procurando traçar, à medida em que caminham, as próprias metas pessoais.

Numa carta de boas-vindas como esta, não conseguiria antecipar realidades para
as quais cada um, cada uma, terá sua própria percepção. Também não gostaria de
me deter em muitos conselhos, pois, tendo superado a dificílima etapa do
vestibular, vocês já demostraram maturidade suficiente. Mas penso que não custa
lhes falar sobre três aspectos que julgo importantes: dar-lhes a conhecer um
pouco da FEEC e de sua história – história muito rica e da qual a partir de
agora vocês fazem parte; apresentar o excelente curso que vocês agora começam;
e, ao menos como sugestão para reflexão, tecer alguns comentários sobre a
expectativa que a sociedade coloca nas pessoas que, como nós, temos ou tivemos o
privilégio de fazer um curso de excelência numa universidade pública da
qualidade da UNICAMP.

Pode-se dizer que a FEEC começou oficialmente suas atividades acadêmicas no
início de 1967, quando ingressou a primeira turma de Engenharia Elétrica da
UNICAMP. Desde então, nossa Escola cresceu em pessoas, recursos e prestígio,
consolidando-se como referência e liderança, tanto no ensino de graduação como
de pós-graduação, ambos fortemente alicerçados na excelência de nossa atividade
em pesquisa. Temos a felicidade de poder contar com um corpo docente de primeira
linha, no qual convivem, em sinergia, a experiência de vários professores que
praticamente começaram a FEEC com o dinamismo de jovens brilhantes que foram
recentemente contratados. Temos também uma infraestrutura que, embora
constantemente necessitada de melhorias, lhes dará as condições adequadas de
estudo, tanto teórico como em laboratório. Mas sabemos que o melhor curso não se
faz apenas com docentes e estrutura; temos consciência de que contamos,
sobretudo, com os melhores estudantes. A partir de agora, vocês também fazem
parte deste corpo discente que é o nosso principal diferencial.

O curso de Engenharia de Computação teve início em 1990, época em que nossa
Escola já contava com grande prestígio, surgindo como uma consequência natural
do bom nível da pesquisa que já realizávamos, à época, nesta área, e das
necessidades de mercado de uma sociedade que começava a orientar-se intensamente
para as tecnologias digitais, que hoje permeiam nossa vida. É um curso que já
nasceu com o selo da excelência e da exigência. Compartilhamos este curso com os
colegas do Instituto de Computação da UNICAMP, unidade de ensino e pesquisa do
mais alto prestígio, na qual vocês também encontrarão um corpo docente
extremamente qualificado. Como as demais engenharias, é um curso que requer uma
base forte, de matemática e física. É importante aproveitar ao máximo esses
primeiros semestres de curso básico, sem se deixar abater por dificuldades que
são naturais, sem perder o ``brilho nos olhos'' desses primeiros dias de
UNICAMP, tendo confiança de que, a despeito de limitações que sempre existem, o
currículo de vocês é harmônico, está bem estruturado e lhes preparará de forma
extremamente adequada para o futuro profissional.

E, juntamente com os estudos, vocês descobrirão a partir de agora a vida
universitária. A Universidade é também um lócus de cultura, debates e,
sobretudo, momento único para se fazer amizades para a vida. Desejo que
aproveitem muito bem cada instante de convivência, que participem com empenho e
alegria das atividades e das entidades estudantis, nas quais vocês descobrirão
um imenso leque de opções para contribuir com a universidade e, a partir dela,
com o país. Mas desejo igualmente que não percam o foco no essencial que é a
própria formação, de modo a não deixar arrefecer seus ideais, nem frustrar as
expectativas que agora não são só de seus familiares, mas de toda a sociedade.
A universidade, pública, gratuita e de excelência, passa por momentos difíceis.
Nossas atividades se sustentam graças ao trabalho de milhões de cidadãos
brasileiros, particularmente do Estado de São Paulo. Esta sociedade tem direito
a nos cobrar eficiência, dedicação e qualidade, a nós, gestores e professores,
mas também aos estudantes, cujo comprometimento ético deve se pautar pela
dedicação ao aprendizado, vencendo a tentação do desânimo, por fomentar o desejo
de saber sempre mais, qualificando-se assim para, no futuro, por meio de seu
trabalho profissional, dar o justo retorno a quem nos financia.

Eu termino com uma citação que gosto muito, é de um autor clássico da
antiguidade grega, Píndaro, que num de seus versos dizia: ``torna-te aquilo que
tu és''. É um chamado do poeta para que o leitor tome consciência de quem é, de
aonde está, e saia assim de um possível momento de alienação ou prostração.
Vocês são hoje estudantes ingressantes do curso de Engenharia de Computação da
UNICAMP. Não é pouca coisa! São certamente orgulho para seus familiares e agora
também patrimônio de nossa Escola. Desejo sinceramente que tenham esta realidade
sempre presente ao longo dos anos em que estiverem aqui. Desde já coloco a
diretoria da FEEC e as coordenações de curso para lhes apoiar neste sentido, em
toda e qualquer dificuldade que possam ter. Sejam bem-vindos, sejam bem-vindas à
FEEC e, sobretudo, sejam muito felizes aqui conosco.

\begin{flushright}
João Marcos Travassos Romano

Diretor da FEEC
\end{flushright}