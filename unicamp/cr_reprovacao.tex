% Este arquivo .tex será incluído no arquivo .tex principal. Não é preciso
% declarar nenhum cabeçalho

\section{Cuidado com CR e Reprovação}

Durante o curso você vai ouvir que se preocupar com seu CR é bobagem, que
estudar para tirar nota não leva a lugar nenhum, que depois de formado não é seu
CR que te colocará no mercado de trabalho, etc. Cuidado, trabalhar numa empresa
não é a sua única opção de vida após formado, e preste atenção, pois ``após
formado'' não significa durante o curso.

Durante o curso você vai ter a possibilidade de participar de várias atividades
acadêmicas e algumas delas vão exigir bom aproveitamento acadêmico d* alun*. Por
exemplo, para se candidatar a monitor de uma disciplina é exigido d* alun* um CR
pelo menos 0,7. Isto significa que sua média de notas das disciplinas cursadas
até aquele momento deve ser maior ou igual a 7,0. Para pleitear uma bolsa de
iniciação científica, onde há concorrência entre alun*s do país todo, também
será exigido bom aproveitamento, assim como para uma bolsa de mestrado. Caso
você não saiba, mestrado faz parte da pós-graduação, ou seja, o seu CR vai te
influenciar até após formado.

Cuidado também com a reprovação. Há instituições, como a FAPESP (Fundação de
Amparo à Pesquisa do Estado de São Paulo), que é a maior fomentadora de
pesquisas do estado de São Paulo e que paga os maiores valores de bolsas do
país, que te excluem de qualquer disputa só por ter uma reprovação no seu
histórico escolar da graduação. Não te exclui oficialmente, mas como é muito
concorrido por ser a melhor pagadora, seu nome vai para o final da lista.

Parece óbvio que quem estuda tira boas notas, mas até você aprender a estudar
como a universidade exige, pode demorar um pouco, e há pessoas que nunca
aprendem.

Há certos períodos (semestres) quando você já estiver mais avançado no curso em
que poderá sentir-se à vontade para desistir de uma disciplina em que esteja
matriculado, deixando para completá-la posteriormente. Quando você fizer isso há
a possibilidade de desistir da disciplina, desmatriculando-se oficialmente dela.
Mas há pessoas que simplesmente deixam de cursar a disciplina, reprovando por
nota e falta e ficando com uma nota baixa em seu histórico. Cuidado com isso,
pode ser frustrante para você no futuro. Por isso, se for desistir de cursar uma
disciplina após matriculado, sempre peça a desistência e tente não reprovar.

Lembre-se de que 4 ou 5 anos é muito tempo, você pode mudar de ideia a qualquer
momento sobre o que pretende fazer no futuro.
