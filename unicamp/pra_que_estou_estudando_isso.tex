% Este arquivo .tex será incluído no arquivo .tex principal. Não é preciso
% declarar nenhum cabeçalho

\section{Para que eu estou estudando isso?}

O ensino médio acabou, você finalmente está livre de todas as inutilidades, como
química orgânica e separação silábica de verbos parnasianos, só vai ver coisas
relevantes para a profissão, e{\dots}

Pimba! HZ291. Pode, Arnaldo?

Primeiro, você precisa saber que a Universidade não é um curso técnico. A ideia
não é só te dar capacitação profissional, mas sim formar pessoas melhores. Para
que um(a) computeir* precisa de contabilidade? Para nada, mas uma pessoa (de
exatas pelo menos) precisa ter uma noção disso.

Outro problema: o que exatamente é ``relevante para a sua profissão''?  A
computação é uma área muito vasta, e a graduação é bem generalista, para te dar
base para escolher. Por exemplo, vai ter gente que nunca mais vai usar
GA/Algelin, mas quem for para a área de computação gráfica vai comer matriz no
café da manhã. Quem garante que no meio do curso você não decida ir para essa
área? Ou ainda, que no seu emprego não te joguem um problema desse tipo?

Se você continuar na universidade, na pós você só terá matérias da sua área, já
que você já sabe o suficiente pra dizer que área é essa. Mas ainda falta muito
chão até lá{\dots}

Para quem é da Engenharia, um problema maior é que, para conseguir o CREA,
existem algumas matérias obrigatórias (embora completamente inúteis, sem
exagero), como Resistência dos Materiais. A Unicamp pode até contrariar essas
orientações, até certo ponto, mas dificilmente *s professor*s concordariam. (Por
outro lado, você poderá construir prédios de até 2 andares. Recomendamos
fortemente que você não faça isso.)

Para quem é da Ciência, o curso não é para formar simples programadores. Vocês
serão mais que isso, serão cientistas, e isso envolve ver coisas além de só
código.
