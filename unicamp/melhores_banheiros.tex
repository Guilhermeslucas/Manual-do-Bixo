% Este arquivo .tex será incluído no arquivo .tex principal. Não é preciso
% declarar nenhum cabeçalho

\section{Melhores banheiros}

Uma das maiores necessidades do ser humano pode ser potencializada se for
realizada num banheiro decente. Portanto, é muito importante que você saiba onde
ir. Alguns dos melhores banheiros da Unicamp são:

\subsubsection{IC-3:} Geralmente estão limpos e utilizáveis. Mas fedem. E em dia
de chuva ficam imundos. Sempre com papel higiênico, é uma boa pedida na hora do
apuro. Exceto nos finais de semana.

\subsubsection{IC-2:} Quase sempre estão limpos e utilizáveis e tem um odor
melhor que os do IC-3. Só precisa tomar cuidado pois às vezes falta papel
higiênico.

\subsubsection{FEEC:} Possui excelentes banheiros escondidos por lá,
principalmente após as reformas de 2013. Procure bem!

\subsubsection{PB:} Os banheiros do segundo e do terceiro andar do Pavilhão
Básico também são bons (especialmente os do terceiro andar, por quase não serem
usados). Só tome cuidado, porque às vezes não tem papel higiênico.

\subsubsection{FE:} A Faculdade de Educação tem poucos banheiros masculinos, mas
estão entre os melhores da Unicamp pelo pouco uso.

\subsubsection{CB:} Estes banheiros ficam escondidos próximo às escadas do CB
(no térreo). Se você tiver sorte de chegar bem após a limpeza, o banheiro estará
em excelentes condições. Porém, na maior parte do tempo ele fica bem sujinho.

\subsubsection{DEQ:} Departamento de Eletrônica Quântica, no IFGW. Dizem que
ninguém os usa.

\subsubsection{DRCC:} Departamento de Raios Cósmicos e Cronologia, no IFGW.  Um
dos melhores banheiros existentes na Unicamp (senão o melhor). Assim como os
banheiros do DEQ, dizem que ninguém os usa.

\subsubsection{DFA:} Departamento de Física Aplicada, no IFGW. Os dois andares
do departamento tem banheiros bons e utilizáveis, mas algumas vezes falta papel
higiênico.

\subsubsection{IMECC:} Todos os três departamentos (andares) do IMECC tem
banheiros bons e utilizáveis. Mas vez ou outra falta papel higiênico.
