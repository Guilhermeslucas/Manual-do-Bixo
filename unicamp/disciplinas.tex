% Este arquivo .tex será incluído no arquivo .tex principal. Não é preciso
% declarar nenhum cabeçalho

\section{Disciplinas}

\subsection{Matrícula}

A Unicamp é muito diferente da sua escolinha onde a tia Gertrudes entregava o
seu horário impresso coloridinho para você colar na capa do seu fichário.  À
exceção do primeiro semestre letivo, no qual você já entra matriculado em todas
as matérias obrigatórias, na Unicamp você vai ter que se virar.  O GDE
(\url{gde.ir}) é uma ferramenta criada por um aluno da Engenharia e adotada pela
DAC que facilita muito o planejamento do seu horário, além de servir como uma
rede social interna.

Peça sempre ajuda * um(a) veteran* quando for montar seu horário. Informe-se
sobre tod*s *s professor*s que oferecem as matérias, se el*s são coxas ou
carrasc*s, se dão aula bem ou mal, se demoram para entregar as notas{\dots} Você
vai poupar muita dor de cabeça. O melhor lugar para essas discussões é o grupo
de e-mail da sua turma, ou então os grupos de Facebook. Pode ter certeza que
haverá grupos assim que você entrar na Unicamp.

Nem sempre você vai conseguir exatamente o que quer na matrícula. Nesses casos,
pode tentar de novo no período de alteração de matrícula, que acontece,
normalmente, próximo ao inícios das aulas. Nesse período é possível pegar uma
matéria que não conseguiu na matrícula normal, permutar a turma de uma matéria
que já se matriculou ou até mesmo desistir de alguma delas, sem nenhum tipo de
penalidade.

\subsection{Cancelamento, Trancamento e Desistência}

Embora praticamente tod*s *s alun*s da Unicamp usem esses três termos
indiscriminadamente, como se fossem sinônimos, para a DAC, esses três termos têm
significados bastante distintos. Aí vai o que cada termo significa:

\subsubsection{Desistência de matrícula em disciplinas} (\url{bit.ly/1BcBsF4}):
Processo que é chamado pel*s alun*s de ``trancamento''.  * alun* não mais cursa
essa disciplina no semestre, tendo de cursá-la em algum semestre posterior (se
for obrigatória) Só é possível desistir uma vez da disciplina e pode-se pedir
desistência até que se tenha passado metade do semestre.
\subsubsection{Cancelamento de matrícula} (\url{bit.ly/1wQBGCq}): Processo em
que * alun* se desliga da Unicamp, por motivo de jubilação, por ter faltado às
duas primeiras semanas do ano de ingresso, por ter sido reprovado em todas as
disciplinas do primeiro ou do segundo semestre de ingresso, por ter sido
expulso, por ter sido aprovado em outra universidade pública (não é permitido
fazer mais do que um curso de universidade pública simultaneamente), ou por
vontade própria do aluno.
\subsubsection{Trancamento de matrícula} (\url{bit.ly/1xWO4SL}): Processo em que
* alun* não cursa qualquer disciplina da Unicamp durante o semestre. * alun* tem
direito a fazer até dois trancamentos de matrícula, em semestres seguidos ou não
e * alun* não pode trancar nenhum dos dois semestres do ano de
ingresso. Desistência de todas as disciplinas configura-se como trancamento. O
trancamento é pedido na DAC, e pode ser pedido até que se tenha transcorrido 2/3
do semestre (geralmente de dezembro até fim de maio para trancamento de primeiro
semestre; e de julho até fim de outubro para trancamento de segundo
semestre). Para cada trancamento, o prazo máximo de integralização é postergado.

\subsection{Mudança de Catálogo}

Pra quem não sabe, ou seja, você mesmo bix*, para cada ano existe um catálogo
correspondente com todas as disciplinas que os alunos devem fazer para se
formar.  Entretanto, esse catálogo não é o mesmo desde o século passado. Como a
computação é uma área muito dinâmica, o curso não pode ficar congelado e as
disciplinas precisam refletir o que o mundo e o mercado atual demandam. Assim,
ao decorrer dos anos ocorrem várias pequenas mudanças, como tirar uma
disciplina, oferecer outra, aumentar créditos, diminuir créditos e assim por
diante, todas feitas com muita discussão pelos professores e alunos.

Como essas mudanças ocorrem num catálogo específico de um ano e só valem a
partir desse, os alunos dos anos anteriores não acompanham essas mudanças e
muitas vezes ficam defasados. Para solucionar isso, é possível mudar de
catálogo.  Como assim? Simples. Se você é um aluno do ano de 2015 você teria que
fazer uma matéria (inútil) chamada FETRANSP. Já em 2017, o catálogo não pede
mais essa disciplina como obrigatória. Então, os alunos de 2015 podem mudar pro
catálogo de 2017 a fim de não fazer mais essa disciplina.

Mas, existem alguns poréns nesse troca-troca. Se você mudar de catálogo, pode
haver matérias a mais que você precise fazer a fim de completar o curso. Também
há o risco de não conseguir equivalência de algumas matérias que são semelhantes
entre si nos catálogos.

Salvo isso, a mudança para catálogos de anos a frente do seu é livre e bem fácil
de conseguir, é só preencher um formulário \url{http://bit.ly/1LHPdyG} e
entregar na DAC. Para mudar para catálogos de anos anteriores (não recomendado)
é um pouco mais difícil, mas não impossível. Basta consultar na DAC.

\subsection{Eletivas e Teste de Proficiência}

A Unicamp oferece a oportunidade de personalizar seu currículo de acordo com seu
interesse por meio das \textbf{disciplinas eletivas}. Ao contrário das
disciplinas obrigatórias, com as eletivas você pode escolher a matéria que vai
cursar. Alguns créditos podem ser cumpridos com qualquer disciplina oferecida
pela Universidade, outros estão restritos a um determinado conjunto. Para mais
detalhes, consulte seu catálogo em \url{bit.ly/1znSCwT}.

Mas não se esqueça de que com um grande poder vem uma grande responsabilidade!
A Unicamp lhe dá liberdade para escolher o melhor jeito de se preparar para seu
futuro, e espera que você saiba o que fazer com essa liberdade. Você pode
socializar com outros cursos, aprender uma língua estrangeira, assistir a
seminários ou obter um certificado de estudos na FEEC ou no IC.

\textbf{Teste de proficiência} é uma prova que permite dispensa de cursar uma
disciplina (desde que você obtenha a nota mínima, é claro). Se você acha que
sabe o suficiente sobre eletromagnetismo, por exemplo, pode tentar a
proficiência de Física Geral III.  Nem todas as disciplinas oferecem o teste, e
você só pode fazê-lo uma vez por disciplina -- e se você já se matriculou na
disciplina e não passou, não pode fazer.  Além disso, fazer o teste de
proficiência também é obrigatório para se matricular nas disciplinas de língua
inglesa e japonesa, independentemente de conhecimento prévio na língua.

Fique ligado no calendário da DAC para não perder as datas de inscrição nos
testes de proficiência! As datas dos testes de línguas são sempre no começo do
ano, diferentes das demais, que são no fim de cada semestre.

Disciplinas eletivas e teste de proficiência estão relacionados porque muitas
pessoas, especialmente nos cursos de computação, fazem proficiência em
disciplinas de línguas, eliminando créditos de eletivas, em alguns casos para
evitar o jubilamento, outros para não ter que passar mais um semestre na
faculdade. Apesar de registrar a familiaridade d* alun* com uma outra língua em
sua integralização, essa prática não enriquece a graduação de nenhum estudante
que faça tal escolha. Além disso, muitos bix*s arrependem-se de terem feito a
prova e perdido preferência na hora de pegar uma disciplina interessante,
podendo até mesmo não conseguir se matricular. Isso acontece porque, depois que
seus créditos de eletivas se esgotam, você começa a puxar matérias
não-obrigatórias como \textbf{extracurriculares} e tem prioridade menor na
atribuição de vagas.

Converse com seus(suas) amig*s e veteran*s para descobrir o melhor jeito de
usufruir dessa liberdade que poucas universidades oferecem! Dificilmente você
não encontrará algo com o qual se identifica ou que não ensine lições
interessantes.

Para mais informações sobre teste de proficiência, acesse: \url{bit.ly/1H3DhaS}.

\subsection{CEL}

O CEL – Centro de Ensino de Línguas – é, como o nome já diz, o orgão responsável
por oferecer aulas de diferentes idiomas a alun*s da Unicamp. Seja como
disciplinas obrigatórias (para a EC, Inglês Instrumental I é uma delas) ou como
eletivas, o CEL possui turmas de várias línguas.

Todas as línguas no CEL oferecem níveis diferentes; assim, se você souber um pouco
de Francês, não precisa começar do início. É só realizar o Teste de Proficiência,
oferecido \textbf{somente} no primeiro semestre de cada ano, e você pode avançar
algumas turmas (ou todas) em uma determinada língua. Também é possível fazer o
teste de proficiência pra eliminar disciplinas (o que a maioria das pessoas acaba
fazendo com Inglês Instrumental). \textbf{Tod*s *s alunos} que quiserem se
inscrever em \textbf{Inglês I (LA112)} ou \textbf{Japonês I (LA111)} devem
\textbf{fazer o teste}, mesmo que não saibam nada da língua, mas ele não é
necessário para matrículas nas outras línguas.

A matrícula em disciplinas do CEL é feita de maneira normal, na DAC, durante o
período de matrícula.

\subsection{Avaliações de professor*s}

Achou que * professor(a) ensinou muito mal? El* falou da vida, do universo e
tudo mais -- menos sobre a disciplina? Foi incoerente? Ou, pelo contrário, achou
o professor(a) o máximo e a sala do CB a oitava maravilha do mundo? Não adianta
xingar nem elogiar no Twitter!

Nas últimas aulas de cada semestre, todos *s professor*s devem disponibilizar um
formulário de avaliação. Esse é o momento para que você possa separar os acertos
dos erros, portanto preencha com seriedade. Os dados serão analisados pelas
Comissões de Graduação de cada unidade e os comentários escritos serão
repassados para o professor.

O IC também faz seu próprio formulário de avaliação da Graduação. Completamente
online, esse formulário é preenchido clicando-se em links disparados para seu
email institucional. O CACo acompanha o processo de envio dos emails para
assegurar todos de que não é associar * alun* à avaliação que ele preencheu.

Além dos formulários, a PRG (Pró-reitoria de Graduação) realiza o Programa de
Avaliação da Graduação no fim de cada semestre. Trata-se de uma pesquisa on-line
semelhante aos formulários de cada unidade, porém unificada para toda a Unicamp
e mais abrangente em suas perguntas.

O GDE (\url{gde.ir}) também tem um sistema de avaliação de professores, cuja
nota costuma ser usada pel*s alun*s como um dos critérios no momento de decidir
com que professor puxar uma matéria.

Durante o semestre, ocorre a Reunião de Avaliação de Curso. A data e o horário
serão divulgados pelas unidades e pelo CACo. Essa é uma oportunidade de passar
para as coordenadorias do curso não só suas impressões sobre professor*s e
disciplinas, mas sobre qualquer assunto relacionado ao curso. Antes da Reunião,
o CACo também promove um PipoCACo de Avaliação de Curso, motivando uma
pré-discussão.

Tenha sempre em mente que a nossa percepção sobre o oferecimento de uma
disciplina não é óbvia para *s professor*s. Preencha todas as avaliações com
sinceridade e use sempre os espaços dedicados a comentários. Além disso, cultive
o hábito de realizar uma avaliação informal d* profess*r no fim de cada semestre
-- mandando um e-mail, por exemplo. O valor deste tipo de avaliação é muito
grande.
